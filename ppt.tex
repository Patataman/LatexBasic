\documentclass{beamer} %documento beamer -> Presentaciones
\usepackage[spanish]{babel} %para indicar que es español
\usepackage[utf8]{inputenc} %para usar la codificación UTF-8 (tíldes por ejemplo)
\usepackage{hyperref} %hipernelaces
\usepackage{animate}
\usepackage{graphicx}

\hypersetup{
    colorlinks,
    citecolor=blue,
    filecolor=blue,
    linkcolor=blue,
    urlcolor=blue
}

\usetheme{Goettingen} %Tematica de colores y estilo
%\usecolortheme{beaver}  %Tematica de colores y estilo
\useinnertheme{rectangles}


\title[Pretend that you're Hercule Poirot] % (optional, only for long titles)
{Pretend that you're Hercule Poirot}
\subtitle{U otra forma de porqué LaTeX mola más que Word}
\author[Daniel A. Rodríguez López] % (optional, for multiple authors)
  {Daniel A. Rodríguez López}
\date[XX-04-2016] % (optional)
  {Jornadas Técnicas XXVIII}
\subject{Jornadas Técnicas XXVIII}

\begin{document}

  \frame{\titlepage}

  \begin{frame}
    \section*{¿Qué es LaTeX?}
    \begin{itemize}
      \item <1->Primero de todo, \textbf{no es un procesador de texto}, olvidaos de Word, LibreOffice y lo que sea que se use en Mac.\\

      \item <2->LaTeX es un sistema de preparación de textos, olvidándonos inicialmente de la parte visual del documento y centrándonos en el texto.
      \item <3->\begin{figure}
        \includegraphics[scale=0.1]{images/imagen3}
      \end{figure}
      \item <4-> En palabras llanas: Latex es un lenguaje para crear documentos de texto desde 0.
    \end{itemize}
  \end{frame}

  \frame{
    Enlaces de interés:
    \begin{itemize}
      \item \textbf{Importante instalarse texlive-full [LINUX]}
      \item \href{https://www.latex-project.org/}{latex-project.org/}
      \item \href{https://en.wikibooks.org/wiki/LaTeX/}{en.wikibooks.org/wiki/LaTeX/}
      \item \href{http://detexify.kirelabs.org/classify.html}{detexify.kirelabs.org/classify.html}
      \item Google es vuestro amigo
    \end{itemize}
  }

\end{document}
