\documentclass[10pt,a4paper,titlepage]{article} %Formato del documento
\usepackage[utf8]{inputenc} %Especificamos que queremos usar utf-8 (ñ o tildes)
%Si no importamos este paquete, las ñ no aparecerían por ejemplo
\usepackage[spanish]{babel}
\usepackage{graphicx}
\usepackage{float}
\usepackage{hyperref} % Permite que los títulos del índice sean enlaces directos a los apartados. Además permite añadir enlaces a internet, referencias...
%%%%%%%%%%%%%%
\hypersetup{
    colorlinks,
    citecolor=black,
    filecolor=black,
    linkcolor=black,
    urlcolor=black
}
%No es realmente necesaria, pero hace que sean más bonitos los enlaces
%%%%%%%%%%%%%%%%
%
%Definimos el título del documento
\title{ \textbf{ \Huge{Latex $>$ Word}} \\ Jornadas Técnicas XXVIII}
\author{
		\begin{tabular}{l}
			\multicolumn{1}{l}{GUL} \\ \hline \\
			Daniel Alejandro Rodríguez López \\
		\end{tabular}
}
%
\begin{document}

\maketitle
\newpage

\section*{Algunas imágenes por aqui, algunas imágenes por allá}
	Para poder incluir imágenes en el documento debemos importar el paquete \textbf{graphicx}. Una vez lo hemos incluido, mediante el comando $\backslash$includegraphics ya podemos incluir imágenes en el documento.

	\begin{center}
		\includegraphics[scale=0.5]{images/imagen1}
	\end{center}

	Una mejor forma de usar las imágenes es mediante la estructura \textbf{$\backslash$begin\{figure\} $<$cosas$>$ $\backslash$end\{figure\}}. Mediante este comando podemos usar una serie de parámetros que son los siguientes
	\begin{itemize}
		\item \textbf{h:} Coloca la imagen aproximádamente en el mismo sitio donde se coloca en el .tex.
		\item \textbf{t:} Coloca la imagen en la parte superior de la hoja.
		\item \textbf{p:} Coloca la imagen en una hoja aparte.
		\item \textbf{b:} Coloca la imagen en la parte inferior de la hoja.
		\item \textbf{!:} Sobreescribe valores por defecto, recomendable su uso.
		\item \textbf{H:} Coloca la imagen en el mismísimo sitio que en el .tex. Requiere el paquete float. \textbf{No es compatible con !}.
	\end{itemize}

	\begin{figure}[H]
		\centering
		\includegraphics[scale=0.3]{images/imagen2}
		\caption{Imagen muy seria}
	\end{figure}

	No hace falta escribir la extensión de la imagen.

\end{document}