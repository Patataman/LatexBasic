\documentclass[10pt,a4paper,titlepage]{article} %Formato del documento
\usepackage[utf8]{inputenc} %Especificamos que queremos usar utf-8 (ñ o tildes)
%Si no importamos este paquete, las ñ no aparecerían por ejemplo
\usepackage[spanish]{babel}
\usepackage{geometry} %Márgenes del documento
\geometry{top=2.54cm, left=3.1cm, right=3.1cm, bottom=2.54cm}

%Definimos el título del documento
\title{ \textbf{ \Huge{Latex $>$ Word}} \\ Jornadas Técnicas XVIII}
\author{
		\begin{tabular}{l}
			\multicolumn{1}{l}{GUL} \\ \hline \\
			Daniel Alejandro Rodríguez López \\
		\end{tabular}
}
%
\begin{document}
%Se genera el título y se realiza un salto de página para que quede bonito
\maketitle
\newpage

	\section{Titulo 1}
	Aqui va texto, y otras cosicas. Estan sencillo como esto. Voy a poner algo con ñ para que se pueda comprobar. Ale, puesto algo con ñ.

	\subsection{Subtitulo 1.1}
		Se pueden hacer varios subsections, hasta 3 subcategorías (1.1.1) con los comandos básicos de Latex. Si creamos nuestro propio comando podemos llegar a muchos más subniveles. 
		\subsubsection{ola k ase}
			xDDD
			%\subsubsubsection{ola k ase 2}  Esto ya no funcionaría
			%	sadsa
\end{document}